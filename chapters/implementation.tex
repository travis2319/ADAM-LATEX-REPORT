% \newpage
\chapter{IMPLEMENTATION}
% \begin{center}
% % \textbf{\LARGE \centering IMPLEMENTATION}\\[1cm]
% \end{center}
\justify
\section{Data Collection from Car OBD Port}
The \textbf{Automotive Diagnostic and Activity Monitoring (ADAM)} system utilizes the car's \textbf{OBD2 (On-Board Diagnostics)} port to collect a wide range of real-time vehicle data, including critical parameters such as engine RPM, vehicle speed, fuel consumption, and engine load. The type and quantity of data available through the OBD2 interface can vary significantly depending on the car's make, model, and the sensors it supports.

\subsubsection{Sensor Variability and Data Challenges}
Each car is equipped with a unique set of sensors, leading to differences in the \textbf{Parameter IDs (PIDs)} that are supported. For instance, while one vehicle might provide detailed oxygen sensor readings, another might lack these advanced diagnostics. The ADAM system adapts to this variability by querying the Supported PIDs (PIDS\_A, PIDS\_B, etc.) to identify which parameters can be accessed for a given vehicle. This ensures effective operation across a broad range of car models.

\subsubsection{Real-Time Data Collection}
Although the system collects data in \textbf{real-time}, there is a slight delay in transmission and processing. Data points are typically retrieved at intervals of \textbf{30 minutes} due to system and hardware limitations, balancing resource efficiency with the need for timely diagnostics. Despite this, the system continuously monitors vehicle performance and updates its diagnostics.

\subsubsection{PIDs And Their Metrics}
The following is a subset of the PIDs that the system may query, along with their descriptions and response values:

\begin{table}[h!]
    \centering
    \caption{Example PIDs and Their Metrics}
    \label{tab:pids_metrics}
    \renewcommand{\arraystretch}{2.5} % Adjust row height (approximately 1 cm with default font size)
    \begin{tabular}{ | c | p{7cm} | p{5cm} | p{3cm} | }
        \hline
        \textbf{PID} & \textbf{Name} & \textbf{Description} & \textbf{Response Value} \\
        \hline
        1  & STATUS               & Status since DTCs cleared              & Special \\  \hline
        4  & ENGINE\_LOAD         & Calculated engine load                 & Unit: Percent \\  \hline
        0C & RPM                  & Engine RPM                             & Unit: RPM \\  \hline
        0D & SPEED                & Vehicle speed                          & Unit: KPH \\  \hline
        10 & MAF                  & Airflow rate (Mass Air Flow)           & Unit: grams/second \\  \hline
        2C & COMMANDED\_EGR       & Commanded Exhaust Gas Recirculation    & Unit: Percent \\  \hline
        5C & OIL\_TEMP            & Engine oil temperature                 & Unit: Celsius \\  \hline
        59 & FUEL\_RAIL\_PRESSURE\_ABS & Absolute fuel rail pressure      & Unit: Kilopascal \\  \hline
    \end{tabular}
\end{table}


\subsection{Adaptability of the ADAM System}
The variability in available PIDs necessitates a flexible system design. The ADAM system is designed to dynamically query the supported PIDs of each vehicle and adjust its data collection and processing pipeline accordingly. This ensures that it can handle everything from basic diagnostics to advanced performance monitoring based on the car's capabilities.

\subsection{Addressing Limitations}
While data is collected in delayed intervals, the insights generated by the system are still valuable for \textbf{diagnostic analysis} and \textbf{predictive maintenance}. Future iterations aim to minimize delays and increase data retrieval speed for a truly real-time experience. By leveraging rich datasets from the OBD2 port and accommodating sensor variability, the ADAM system provides robust and customizable diagnostics for various vehicles.

\section{Data Transfer to Go Server}
After data is collected from the car’s OBD port, it is sent to a \textbf{Go server}. This server acts as middleware to handle data flow efficiently from the Python script to backend systems, ensuring seamless data exchange for real-time monitoring and analysis.

\section{Data Storage in PostgreSQL Database}
The collected data is stored in a \textbf{PostgreSQL (PGSQL)} database for future processing and analysis. This database serves as a central repository for storing vehicle data, including various parameters like engine temperature, fuel consumption, speed, etc. The structured format allows for efficient querying and data retrieval.

\section{Data Analysis Using FastAPI and Machine Learning}
Once stored in PostgreSQL, \textbf{FastAPI} is used to access and process this data for various analyses. Machine learning models are applied to perform several tasks:

\begin{enumerate}
    \item \textbf{Predictive Maintenance}: ML algorithms predict potential failures of car components by analyzing historical and real-time data.
    \item \textbf{Driver Behavior}: The system tracks aspects of driver behavior such as acceleration and braking patterns.
    \item \textbf{Emission Compliance}: It monitors emission-related parameters to ensure compliance with environmental regulations.
    \item \textbf{Engine Health Prediction}: Real-time data analysis predicts potential issues like overheating or mechanical failures.
\end{enumerate}

\section{Chatbot for User Interaction}
An integrated chatbot allows users to interact with the system regarding their car's performance metrics. It can answer queries such as:
\begin{itemize}
    \item \textbf{Current vehicle health} (e.g., engine status, fuel efficiency)
    \item \textbf{Upcoming maintenance tasks}
    \item \textbf{Emission compliance}
    \item \textbf{Engine health status and alerts}
\end{itemize}

This feature provides users with a conversational interface for obtaining insights without complex navigation through menus.

\section{Data Visualization via Mobile App}
The ADAM system presents vehicle diagnostics through a mobile app interface that visualizes key performance metrics such as:
\begin{itemize}
    \item \textbf{Maintenance schedules and alerts}
    \item \textbf{Driver behavior analytics}
    \item \textbf{Engine health status and predictions}
    \item \textbf{Emission compliance status}
\end{itemize}

This user-friendly app enables car owners to track their vehicle's performance, receive maintenance notifications, and monitor real-time data for optimal vehicle health management.
